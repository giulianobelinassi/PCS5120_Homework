\documentclass[12pt]{article}

\usepackage{sbc-template}

\usepackage{graphicx,url}

%\usepackage[brazil]{babel}   
\usepackage[latin1]{inputenc}  

\usepackage{amsmath}
\usepackage{amssymb} 
\usepackage{mathtools}

\usepackage{algorithm}
\usepackage[noend]{algpseudocode}

\usepackage{hyperref}
 
\newcommand{\Cfield}{\mathbb{C}}
\newcommand{\Rfield}{\mathbb{R}}

\newcommand{\norm}[1]{\left\lVert#1\right\rVert}

\sloppy

\title{PCS5120 Homework}

\author{Giuliano A. F. Belinassi\inst{1}}


\address{Institute of Mathematics and Statistics (IME) -- University of S�o Paulo
  (USP)\\
  Rua do Mat�o, 1010 -- S�o Paulo -- SP -- Brazil
}

\begin{document} 

\maketitle

%\begin{abstract}

%The Boundary Element Method requires a geometry discretization to execute simulations, 
%and it can be used to analyze the 3D stationary behavior of wave propagation in the soil. 
%Such discretization involves generating two high computational power demanding matrices, 
%and this article demonstrates how Graphical Processing Units (GPU) were used to accelerate 
%this process. In an experiment with 4000 Mesh elements and 1600 Boundary elements, 
%a speedup of 107$\times$ was obtained with a GeForce GTX980. 

%\end{abstract}
     
%\section{Introduction}

In various research fields where numerical linear algebra is required eighter because of 
its facility or inexisence of analytical solutions, can take advantage of packages
such as Lapack or BLAS. One major concern about these libraries is its 
performance, subject discussed in this report. Here we target the routine designed to multiply 
two matrices in single precision floating point (\texttt{SGEMM}).

Althrough implementing a matrix-matrix multiplication seems trivial by its concept, it is 
fairly difficult to provide an efficient code because of various reasons, such as cache usage. 
The use of techniques such as block matrix multiplication can yield better results due 
to a better cache usage, but a question that can be asked from this approach is what is the 
block size that maximizes performance?

\section{The database}

We used the database "\textit{SGEMM GPU kernel performance Data Set}" provided by \textit{UCI machine 
learning repository}\footnote{\url{https://archive.ics.uci.edu/ml/datasets/SGEMM+GPU+kernel+performance}}. 
Briefly, it has timings of multiplication of two matrices in \textit{ms}, each 
one of size $2048\times2048$, using a combination of $14$ parameters, totalizing $241601$ lines in the database. 

\section{Data analysis}

We used Orange in our analysis. Our first objective was to find the distribuition of the average 
of the four executions per sample to check possible improvements or deterioration of the performance.
For creating a column with the average of four executions, we used the Orange's Feature Constructor. 
Unfortunately, there is no average function implemented here, so we had to calculate such function by 
the definition $\left(\frac{1}{n} \sum{x_i} \right)$. 

Once we had the average column, we plotted the distribution, as illustrated by \ref{fig:freq}. 
Notice that most of the averages concentrate under 200ms, and there are some data around 2400ms. 
Such high timing can be caused by the parameters itself or be an outlier because there were other 
programs running on the computer.

Once we got the distribuition of timings, we focused on what parameters yielded best results. 
Orange's Data Table let us find the combination of parameters because how straightfoward the 
table sorting feature is implemented.

Maybe for pratical reasons, the showed results is enough to provide a setup for an efficient 
implementation, but we can explore the provided data in order to create projections for perhaps
even better timings.


\begin{figure}[ht]
\centering
\includegraphics[scale=0.35]{freq.png}
\caption{Distribution of the average of four samples per parameter.}
\label{fig:freq}
\end{figure}

\begin{table}[]
\centering
\caption{Combination of parameters that yielded the 5\# best results}
\label{my-label}
\resizebox{\columnwidth}{!}{%
\begin{tabular}{|l|l|l|l|l|l|l|l|l|l|l|l|l|l|l|}
\hline
MWG & NWG & KWG & MDIMC & NDIMC & MDIMA & NDIMB & KWI & VWM & VWN & STRM & STRN & SA & SB & Mean (ms) \\ \hline
16  & 16  & 16  & 8     & 8     & 8     & 8     & 2   & 1   & 1   & 0    & 0    & 0  & 0  & 116,37    \\ \hline
16  & 16  & 16  & 8     & 8     & 8     & 8     & 2   & 1   & 1   & 0    & 0    & 0  & 1  & 78,705    \\ \hline
16  & 16  & 16  & 8     & 8     & 8     & 8     & 2   & 1   & 1   & 0    & 0    & 1  & 0  & 80,565    \\ \hline
16  & 16  & 16  & 8     & 8     & 8     & 8     & 2   & 1   & 1   & 0    & 0    & 1  & 1  & 86,6375   \\ \hline
16  & 16  & 16  & 8     & 8     & 8     & 8     & 2   & 1   & 1   & 0    & 1    & 0  & 0  & 118,6625  \\ \hline
\end{tabular}%
}
\end{table}

\section{Conclusions}


%The current implemented code have limitations. First, there is no logic to construct both $H$ and $G$ by blocks to create several 
%GPU kernels. Second, there is also no logic to compute both $\texttt{Ghmatecd\_Nonsingd}$ and $\texttt{Ghmatecd\_Sing\_de}$ in 
%parallel with respect to each other. The usage of GPUs in the singular case can also be analyzed.


%\begin{figure}[ht]
%\centering
%\includegraphics[width=.5\textwidth]{fig1.jpg}
%\caption{A typical figure}
%\label{fig:exampleFig1}
%\end{figure}
%
%\begin{figure}[ht]
%\centering
%\includegraphics[width=.3\textwidth]{fig2.jpg}
%\caption{This figure is an example of a figure caption taking more than one
%  line and justified considering margins mentioned in Section~\ref{sec:figs}.}
%\label{fig:exampleFig2}
%\end{figure}


\bibliographystyle{sbc}
\bibliography{sbc-template}

\end{document}
\grid
